\documentclass{article}
\usepackage[T1]{fontenc}
\usepackage[utf8]{inputenc}
\usepackage{lmodern}
\usepackage{textcomp}
\usepackage{tikz}
\usepackage{pgfplots}
\usepackage{ragged2e}
\usepackage{geometry}
\geometry{
top=5mm,
left=15mm,
right=15mm,}
\pgfplotsset{compat=newest}

\title{Photoelectric Effect}
\date{}
\author{Rosie Bartlett}

\begin{document}
\maketitle
\begin{enumerate}
\item \begin{enumerate}
\item Photoelectric emission from a metal surface is the emission of electrons from a metal surface due to the photoelectric effect.

\item Since for each metal the valence electrons are attracted by different amounts by the nucleus, each photon that hits the metal must have a minimum amount of energy, called the work function $\phi$, to remove the valence electron. $\phi$ is different for each type of metal since it is dependant on the attraction between the atom and nucleus.

\end{enumerate}

\item \begin{enumerate}
\item \begin{enumerate}
\item $E=hf=450\times10^{-9}\times h\approx3.0\times10^{-40}J$

\item $E=hf=1500\times10^{-9}\times h\approx9.9\times10^{-40}J$

\end{enumerate}

\item Since $E=hf$, and $v=f\lambda$, as $\lambda$ increases, $f$ must decrease, which means the energy of the photon must also decrease. When $\lambda\in\langle 450\times10^{-9}, 650\times10^{-9} \rangle$ then at some point $E=\phi$, meaning that at a higher wavelength than when that occurs, no electrons can be emitted because the photons do not have enough energy to move them.

\end{enumerate}

\item Using $f=\frac{e\phi_{eV}}{h}$:
\begin{enumerate}
\item Caesium, potassium

\item Silver

\item Caesium

\item 0. A photon with a wavelength of 300nm has less energy than $\phi$

\item 0.18V

\end{enumerate}

\item 1.3V

\item 3.7$\times10^{-25}$ms$^{-1}$

\item \begin{enumerate}
\item 3.1$\times10^{-19}$J

\item $\phi=$1.6$\times10^{-19}$J

\item $f_0=$2.5$\times10^{14}$Hz

\end{enumerate}

\end{enumerate}
\end{document}