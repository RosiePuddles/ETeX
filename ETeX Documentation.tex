\documentclass{article}
\usepackage[T1]{fontenc}
\usepackage[utf8]{inputenc}
\usepackage{lmodern}
\usepackage{textcomp}
\usepackage{hyperref}
\usepackage{geometry}
\usepackage{minted}
\usepackage{multicol}
\usepackage{color, soul}
\hypersetup{colorlinks,
citecolor = blue,
filecolor = blue,
linkcolor = blue,
urlcolor = blue
}
\geometry{}


\title{ETeX Documentation\\\large V0.1}
\date{}
\author{RosiePuddles}

\begin{document}
\maketitle
\tableofcontents
\newpage

\section{Preface}\label{sec:preface}
This package is designed to allow the user to generate LaTeX  files and associated pdf files in a more user friendly way. Please note, however, this package is still currently heavily in development, and things will go wrong. Any bugs can be reported on the href{https:/github.comRosiePuddlesETeX_from_pythonissues}{issues page} of the GitHub repository. You can request any features you cannot find and want adding to the package. Having said that, I hope you find this package useful and fairly easy to use as intended.
Please note that every class inherits from the verb|_main| class unless specified otherwise. Each class that inherits from verb|_main| may overwrite methods defined in the verb|_main| class. If a class does overwrite a predefined method this will be documented, otherwise there will be no specific documentation if the method is inherited.
\section{Main Classes}\label{sec:main_classes}
This section is for the documentation of the classes contained within verb|ETeX|.
Each of the classes in this section, except for the verb|Document| class, inherit from the verb|_main| class\footnote{See \autoref{subsubsec:child_classes} for a full list of classes that directly or indirectly inherit from the \_main class.}. As such please be aware that when looking for documentation on a certain method that the method may be inherited and documentation will be contained within the verb|_main| class section.
\subsection{Document}\label{subsec:document}
\begin{minted}{python}
class Document:
	def __init__(self, *args, **kwargs) -> None
\end{minted}
The verb|Document| class is the main class used in ETeX. It handles all tex codegeneration, and contains all information about the document as well as the actual contents themself.
\subsubsection{generate\_TeX method}\label{subsubsec:generate_tex_method}
\begin{minted}{python}
Document.generate_TeX(self, _compile: bool = True, **kwargs) -> str
\end{minted}
The verb|generate_TeX| function is used to firstly generate the .tex file which is then compiled and the resulting .pdf file is opened. The parameter verb|_compile| is set to verb|False| by default. If it is changed to verb|True|, then only the .tex file will be generated, but not compiled. For the verb|kwargs|, you can pass in verb|debug=True| to see the logs from pdflatex as it compiles the .tex file.
The output .tex file name will be a formatted version of the value for the title given on instantiation of a new instance of the verb|Document| class. Any of the following characters are removes:\begin{itemize}
\item $
\item %
\item /
\item textbackslash
\end{itemize}
Full stops are also formatted and turned into underscores. The resulting formatted filename is then used for all of the resulting output files.
\subsubsection{add method}\label{subsubsec:add_method}
\begin{minted}{python}
Document.add(self, item: _main) -> None
\end{minted}
The verb|add| method adds items to the document. The added item must inherit from the verb|_main| class. Only items added to the verb|Document| class instance will be included in the final document.
\subsubsection{new\_section method}\label{subsubsec:new_section_method}
\begin{minted}{python}
Document.new_section(self, title: str, _type: int = 0) -> None
\end{minted}
The verb|new_section| function is used to add a new section to a verb|Document| class instance. The verb|_type| argument is used to identify the type of section with 0 being a section, 1 being a subsection, and 2 being a subsubsection.
\subsection{\_main}\label{subsec:_main}
The verb|_main| class is the base class for all other classes the user interfaces with and provides several important functions and alterations to base functions that are used throughout.
\subsubsection{Child classes}\label{subsubsec:child_classes}
This section provides a list of all the different child classes of the verb|_main| class. This is split up into two parts. Those that directly inherit from the verb|_main| class, and those that inherit from the verb|_main| class through inheriting from the verb|_holder| class.
Classes that inherit from verb|_main|:\begin{multicols}{2}\begin{itemize}
\item verb|Text|
\item verb|Footnote|
\item verb|Equation|
\item verb|line|
\item verb|plot|
\item verb|coordinates|
\item verb|axis|
\item verb|Code|
\item verb|Chemical|
\item verb|ChemEquation|
\end{itemize}
\end{multicols}Classes that inherit from verb|_holder|:\begin{multicols}{2}\begin{itemize}
\item verb|Columns|
\item verb|List|
\item verb|Group|
\end{itemize}
\end{multicols}
\subsection{\_holder}\label{subsec:_holder}
\begin{minted}{python}
class _holder(_main):
	def __init__(self, packages) -> None
\end{minted}
The verb|_holder| class is a second base class that inherits from the verb|_main| class. The class adds the verb|add| method and allows for child classes to have class instances added to them. For a full list of classes that inherit from verb|_holder| see autoref{subsubsec:child_classes}.
\subsubsection{add method}\label{subsubsec:add_method001}

\subsection{Text}\label{subsec:text}
\begin{minted}{python}
class Text(_main):
	def __init__(self, text: str, align: str = None) -> None
\end{minted}
The verb|Text| class is the class used for the handling of text inside of ETeX. The class contains some general string formatting features allowing for \textbf{bold}, \textit{italic}, \hl{highlighted}, and \underline{underlined} text inside of the document. To read more on this see autoref{subsubsec:inbuilt_formatting}. The text can also be aligned to either the left, center, or right using the verb|align| argument. This will only apply to the current verb|Text| class instance and will not be applied to any subsequent instances of the class.
\subsubsection{Inbuilt formatting}\label{subsubsec:inbuilt_formatting}
To format a string in ETeX, you use the * and \\
{} characters. The following table shows the formatting character and the relevant format.\\

\begin{center}
\begin{tabular}{| c | l |}
\hline
Formatting character & Associated formatting \\ \hline
* & \textbf{Bold} \\
** & \textit{Italic} \\
$sim$ & \hl{Highlight} \\
$simsim$ & \underline{Underline} \\
\hline
\end{tabular}
\end{center}

\subsubsection{Extra formatting}\label{subsubsec:extra_formatting}
Withing the text environment regular LaTeX  commands can be used. Some useful examples are given below:\begin{itemize}
\item {textbackslash}verb$mid${foo}$mid$ produces text in a code-like font as seen below:\\
erb|foo|
\item $The $ character allows you to write inline maths equations such as the example below:\\
$2x+y^{}3=-1$ $rightarrow 2x+y^3=-1$
\end{itemize}

\subsection{List}\label{subsec:list}
\begin{minted}{python}
class List(_holder):
	def __init__(self, list_type: str = 'numbered', items: list = None) -> None
\end{minted}
The verb|List| class is used to created lists inside of ETeX. These list can be either a numbered list or a bullet point list through the use of the verb|list_type| argument. The list can also be initialised with items already inside of it, so long as the items inherit from the verb|_main| class. The list can also be left empty upon initialisation and later on have items added to it using the verb|add| function.
\subsubsection{List types}\label{subsubsec:list_types}
To change the type of list, you can use the verb|list_type| argument, which takes in a string of wither verb|numbered| or verb|bullet|, which correspond to a numbered list, or a bullet point list.
\subsection{Group}\label{subsec:group}
\begin{minted}{python}
class Group(_holder):
	def __init__(self, items: list = None) -> None
\end{minted}
The verb|Group| class is a holding class used for storing other classes. The primary use for this class is alongside lists. When an item is added to a list it is added as a new item, however if the user wants to add several different classes to a list as the same point they can put all the items into a verb|Group| class and add that to the list.
\subsection{Columns}\label{subsec:columns}
\begin{minted}{Python}
class Columns(_holder):
	def __init__(self, columns: int, items: list = None, unbalanced: bool = False) -> None
\end{minted}
The verb|Columns| class is used to add columns to the document. It is similar to the verb|Group| class in that it stores classes to be contained within it's formatting. Only items added to the class will be put into columns. To make the columns unbalances, i.e. with the contents not spread out equally over all the columns, you can change the verb|unbalanced| argument to verb|True|.
\section{Maths Classes}\label{sec:maths_classes}

\section{Plotting Classes}\label{sec:plotting_classes}
This section is for classes contained within verb|ETeX.maths.plots|. All classes inherit from verb|_main| unless stated otherwise.
\subsection{Axis}\label{subsec:axis}
\begin{minted}{python}
class Axis(_main):
	def __init__(self, *args, **kwargs) -> None
\end{minted}
The verb|Axis| class is the handler for all plots. It is centre justified. Within the verb|**kwargs| argument there are a large number of parameters that we can pass in. these are listed below:\begin{itemize}
\item verb|title: str|This is the title of the axis and is positioned centre justified above the axis
\item verb|width: int or float|This is the width of the axis. This is measured in cm.
\item verb|height: int or float|This is the height of the axis. This is measured in cm.
\item Min and max values:
These correspond to the minimum and maximum $x$ and $y$ values on the axi. If none are specified the minimum or maximum values of the plots contained within the axis will be used instead. The following options are available:\begin{itemize}
\item verb|xmin: int or float|
\item verb|xmax: int or float|
\item verb|ymin: int or float|
\item verb|ymax: int or float|
\end{itemize}

\item Axis labels:
These correspond to the $x$ and $y$ axis labels. The following options are available:\begin{itemize}
\item verb|xlab: str|
\item verb|ylab: str|
\end{itemize}

\item verb|samples: int|This is the number of samples used for plotting functions. By default it is set to 100.
\item verb|showTickMarks: bool|This bool controls weather or not tick marks are shown on the $x$ and $y$ axes. This is set to verb|True| by default.
\item verb|clip: bool|This bool controls weather or not the plots can be clipped to fit within the axis. This is set to verb|False| by default.
\end{itemize}

\subsubsection{add\_plot method}\label{subsubsec:add_plot_method}
\begin{minted}
Axis.add_plot(self, new_plot: plot or coordinates) -> None
\end{minted}
The verb|add_plot| method adds a plot to the current verb|Axis| instance. The plot must be an instance of either a verb|plot| or verb|coordinates| class.
\subsection{Plot}\label{subsec:plot}
\begin{minted}{python}
class Plot(_main)
	def __init__(self, function: str, domain: tuple = None, color=None, name: str = None) -> None
\end{minted}

\section{Chemistry Classes}\label{sec:chemistry_classes}
\end{document}