\documentclass{article}
\usepackage[T1]{fontenc}
\usepackage[utf8]{inputenc}
\usepackage{lmodern}
\usepackage{textcomp}
\usepackage{hyperref}
\usepackage{geometry}
\usepackage{xcolor}
\usepackage{listings}
\usepackage{color, soul}
\usepackage{multicol}
\hypersetup{colorlinks,
citecolor = blue,
filecolor = blue,
linkcolor = blue,
urlcolor = blue
}
\definecolor{codegreen}{rgb}{0,0.6,0}
\definecolor{codefunc}{rgb}{0.9.1,0.471,0.2}
\lstdefinestyle{mystyle}{
commentstyle=\color{codegreen},
keywordstyle=\color{codefunc},
stringstyle=\color{codegreen},
basicstyle=\ttfamily\footnotesize
}\lstset{style=mystyle}
\geometry{}


\title{ETeX Documentation\\\large V0.1}
\date{}
\author{RosiePuddles}

\begin{document}
\maketitle
\tableofcontents
\newpage

\section[Preface]{Preface}
This package is designed to allow the user to generate \LaTeX  files and associated pdf files in a more user friendly way. Please note, however, this package is still currently heavily in development, and things will go wrong. Any bugs can be reported on the \href{https:/github.comRosiePuddlesETeX_from_pythonissues}{issues page} of the GitHub repository. You can request any features you cannot find and want adding to the package. Having said that, I hope you find this package useful and fairly easy to use as intended.\\
Please note that every class inherits from the \verb|_main| class unless specified otherwise. Each class that inherits from \verb|_main| may overwrite methods defined in the \verb|_main| class. If a class does overwrite a predefined method this will be documented, otherwise there will be no specific documentation if the method is inherited.
\section[Main Classes]{Main Classes}
This section is for the documentation of the classes contained within \verb|ETeX.main|.\\
Each of the classes in this section, except for the \verb|Document| class, inherit from the \verb|_main| class. As such please be aware that when looking for documentation on a certain method that the method may be inherited and documentation will be contained within the \verb|_main| class section.
\subsection[Document]{Document}
\lstset{language=python}
\begin{lstlisting}
class Document(*args, **kwargs) -> None
\end{lstlisting}
The \verb|Document| class is the main class used in ETeX. It handles all tex codegeneration, and contains all information about the document as well as the actual contents themself.
\subsubsection[generate\_TeX method]{generate\_TeX method}
\lstset{language=python}
\begin{lstlisting}
Document.generate_TeX(self, _compile: bool = True, **kwargs) -> str
\end{lstlisting}
The \verb|generate_TeX| function is used to firstly generate the .tex file which is then compiled and the resulting .pdf file is opened. The parameter \verb|_compile| is set to \verb|False| by default. If it is changed to \verb|True|, then only the .tex file will be generated, but not compiled. For the \verb|kwargs|, you can pass in \verb|debug=True| to see the logs from pdflatex as it compiles the .tex file.
The output .tex file name will be a formatted version of the value for the title given on instantiation of a new instance of the \verb|Document| class. Any of the following characters are removes:\begin{itemize}
\item \$
\item \%
\item /
\item \textbackslash
\end{itemize}
Full stops are also formatted and turned into underscores. The resulting formatted filename is then used for all of the resulting output files.
\subsubsection[new\_section method]{new\_section method}
\lstset{language=python}
\begin{lstlisting}
Document.add(self, title: str, _type: int = 0) -> None
\end{lstlisting}
The \verb|new_section| function is used to add a new section to a \verb|Document| class instance. The \verb|_type| argument is used to identify the type of section with 0 being a section, 1 being a subsection, and 2 being a subsubsection.
\subsection[Text]{Text}
\lstset{language=python}
\begin{lstlisting}
class Text(self, text: str, align: str = None) -> None
\end{lstlisting}
The \verb|Text| class is the class used for the handling of text inside of ETeX. The class contains some general string formatting features allowing for \textbf{bold}, \textit{italic}, \hl{highlighted}, and \underline{underlined} text inside of the document. To read more on this see section 2.2.1. The text can also be aligned to either the left, center, or right using the \verb|align| argument. This will only apply to the current \verb|Text| class instance and will not be applied to any subsequent instances of the class.
\subsubsection[String formatting]{String formatting}
To format a string in ETeX, you use the * and \~{} characters. The following table shows the formatting character and the relevant format.\\
\begin{center}
\begin{tabular}{| c | l |}
\hline
Formatting character & Associated formatting \\ \hline
* & \textbf{Bold} \\
** & \textit{Italic} \\
\~{} & \hl{Highlight} \\
\~{}\~{} & \underline{Underline} \\
\hline
\end{tabular}
\end{center}

\subsection[List]{List}
\lstset{language=python}
\begin{lstlisting}
class List(self, list_type: str = 'numbered', items: list = None) -> None
\end{lstlisting}
The \verb|List| class is used to created lists inside of ETeX. These list can be either a numbered list or a bullet point list through the use of the \verb|list_type| argument. The list can also be initialised with items already inside of it, so long as the items inherit from the \verb|_main| class\footnote{See section 2.5.1 for a full list of classes that directly or indirectly inherit from the \_main class.}. The list can also be left empty upon initialisation and later on have items added to it using the \verb|add| function.
\subsubsection[List types]{List types}
To change the type of list, you can use the \verb|list_type| argument, which takes in a string of wither \verb|numbered| or \verb|bullet|, which correspond to a numbered list, or a bullet point list.
\subsection[Group]{Group}
\lstset{language=python}
\begin{lstlisting}
class Group(self, items: list = None) -> None
\end{lstlisting}
The \verb|Group| class is a holding class used for storing other classes. The primary use for this class is alongside lists. When an item is added to a list it is added as a new item, however if the user wants to add several different classes to a list as the same point they can put all the items into a \verb|Group| class and add that to the list.
\subsection[Columns]{Columns}
\lstset{language=Python}
\begin{lstlisting}
class Columns(self, columns: int, items: list = None, unbalanced: bool = False) -> None
\end{lstlisting}
The \verb|Columns| class is used to add columns to the document. It is similar to the \verb|Group| class in that it stores classes to be contained within it's formatting. Only items added to the class will be put into columns. To make the columns unbalances, i.e. with the contents not spread out equally over all the columns, you can change the \verb|unbalanced| argument to \verb|True|.
\subsection[\_main]{\_main}
The \verb|_main| class is the base class for all other classes the user interfaces with and provides several important functions and alterations to base functions that are used throughout.
\subsubsection[Child classes]{Child classes}
This section provides a list of all the different child classes of the \verb|_main| class:\begin{multicols}{2}\begin{itemize}
\item \verb|Text|
\item \verb|Footnote|
\item \verb|Columns|
\item \verb|Equation|
\item \verb|List|
\item \verb|Group|
\item \verb|line|
\item \verb|plot|
\item \verb|coordinates|
\item \verb|axis|
\item \verb|Code|
\item \verb|Chemical|
\item \verb|ChemEquation|
\end{itemize}
\end{multicols}
\subsubsection[generate\_TeX method]{generate\_TeX method}
\lstset{language=python}
\begin{lstlisting}
_main.generate_TeX(self, *args, **kwargs) -> str
\end{lstlisting}
The \verb|generate_TeX| method generates raises an exception if run. All classes that inherit from \verb|_main| will overwrite this method with their own method to generate their unique \LaTeX  code.
\subsubsection[add method]{add method}
\begin{lstlisting}
Columns.add(self, item: _main) -> None:
	self.items.append(item)
	self.add_super(item.packages)
\end{lstlisting}
The \verb|add| method of the Columns class adds a given item that inherits from the \verb|_main| class to the class's list of items inside it. It also adds any packages required by the added item to it's own packages list to then eventually be passed into the \verb|Document| class instance and be added to the .tex file.
\section[Maths Classes]{Maths Classes}

\section[Plotting Classes]{Plotting Classes}
This section is for classes contained within \verb|ETeX.maths.plots|. All classes inherit from \verb|_main| unless stated otherwise.
\subsection[Axis]{Axis}
\lstset{language=python}
\begin{lstlisting}
class Axis(self, *args, **kwargs) -> None
\end{lstlisting}
The \verb|Axis| class is the handler for all plots. It is centre justified. Within the \verb|**kwargs| argument there are a large number of parameters that we can pass in. these are listed below:\begin{itemize}
\item \verb|title: str|\\This is the title of the axis and is positioned centre justified above the axis
\item \verb|width: int or float|\\This is the width of the axis. This is measured in cm.
\item \verb|height: int or float|\\This is the height of the axis. This is measured in cm.
\item Min and max values:\\
These correspond to the minimum and maximum $x$ and $y$ values on the axi. If none are specified the minimum or maximum values of the plots contained within the axis will be used instead. The following options are available:\begin{itemize}
\item \verb|xmin: int or float|
\item \verb|xmax: int or float|
\item \verb|ymin: int or float|
\item \verb|ymax: int or float|
\end{itemize}

\item Axis labels:\\
These correspond to the $x$ and $y$ axis labels. The following options are available:\begin{itemize}
\item \verb|xlab: str|
\item \verb|ylab: str|
\end{itemize}

\item \verb|samples: int|\\This is the number of samples used for plotting functions. By default it is set to 100.
\item \verb|showTickMarks: bool|\\This bool controls weather or not tick marks are shown on the $x$ and $y$ axes. This is set to \verb|True| by default.
\item \verb|clip: bool|\\This bool controls weather or not the plots can be clipped to fit within the axis. This is set to \verb|False| by default.
\end{itemize}

\subsubsection[add\_plot method]{add\_plot method}
\begin{lstlisting}
Axis.add_plot(self, new_plot: plot or coordinates) -> None
\end{lstlisting}
The \verb|add_plot| method adds a plot to the current \verb|Axis| instance. The plot must be an instance of either a \verb|plot| or \verb|coordinates| class.
\section[Chemistry Classes]{Chemistry Classes}
\end{document}